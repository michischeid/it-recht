% wa-konfiguration-englisch
%
% 2012-12-13
% 
% Diese Datei wird f\"ur die Sprachoption englisch verwendet, d. h.  
% \newcommand{\seWaSprache}{englisch}
%
%
% In dieser Datei k\"onnen Neudefinitionen vorgenommen werden f\"ur:
% -- Verzeichnisse
% -- Unter-/\"Uberschriften von Abbildungen, Tabellen und Listings
% -- Querverweise innerhalb des Textes

%
%  Konfiguration der verschiedenen Verzeichnisse
%
%  abstandEintrag: Wert wird mit \baselineskip multipliziert
%

%
%  Abbildungsverzeichnis
%
\seKonfigurationAbb[
verzeichnisname=List of Figures,
labeltextLinks=, % kein Text links;
%labeltextRechts=:,
labelbreite=1cm,
%labeleinzug=1cm,
%abstandEintrag=1,
%newpage=ja,
%pnumwidth=20mm,
%dotsep=1000,
%tocrmarg=4.5cm,
%abstandVerzeichnis=-1mm
]

%
% LIstingverzeichnis
%
\seKonfigurationPrg[
verzeichnisname=List of Program Listings,
labeltextLinks=,
%labeltextRechts=:,
labelbreite=1cm,
%labeleinzug=2cm,
%abstandEintrag=1,
%newpage=ja,
%%pnumwidth=20mm,
%dotsep=1000,
%tocrmarg=4.5cm,
%abstandVerzeichnis=-10mm
]


% 2013-01-26
%
% Algorithmenverzeichnis
%
\seKonfigurationAlg[
verzeichnisname=List of Algorithms,
labeltextLinks=,
%labeltextRechts=:,
labelbreite=1cm,
%labeleinzug=2cm,
%abstandEintrag=1,
%newpage=ja,
%pnumwidth=20mm,
%dotsep=1000,
%tocrmarg=4.5cm,
%abstandVerzeichnis=-10mm
]


%
% Tabellenverzeichnis
%
\seKonfigurationTab[
verzeichnisname=List of Tables,
labeltextLinks=,
%labeltextRechts=:,
labelbreite=1cm,
%labeleinzug=0.5cm,
%abstandEintrag=1,
%newpage=ja,
%pnumwidth=20mm,
%dotsep=1000,
%tocrmarg=4.5cm,
%abstandVerzeichnis=-10mm
]

%
% Abk\"urzungsverzeichnis
%
\seKonfigurationAbk[
verzeichnisname=List of Abbreviations,
%labelbreite=3cm,
%labeleinzug=0.5cm,
%abstandEintrag=1,
%newpage=ja,
%abstandVerzeichnis=-10mm
]

%
% Symbolverzeichnis
% 
\seKonfigurationSym[
verzeichnisname=List of Symbols,
%labelbreite=4cm,
%labeleinzug=3.5cm,
%abstandEintrag=1,
%newpage=ja,
%abstandVerzeichnis=-10mm
]

%
% Glossar
%
\seKonfigurationGlo[
verzeichnisname=Glossary,
%abstandEintrag=0,
]



% (eventuelle) Neudefinition f\"ur die Unter-/\"Uberschriften von Abbildungen, Tabellen und Listings
%
%
\renewcommand{\seCaptionNameAbbildung}{Figure}
\renewcommand{\seCaptionNameTabelle}{Table}
\renewcommand{\seCaptionNameProgramm}{Listing}
\renewcommand{\seCaptionNameAlgorithmus}{Algorithm}


% % (eventuelle) Neudefinition f\"ur Querverweise innerhalb des Textes
%
%
%
\renewcommand{\seQuerverweisSeite}{page}
\renewcommand{\seQuerverweisAbbildung}{figure}
\renewcommand{\seQuerverweisTabelle}{table}
\renewcommand{\seQuerverweisProgramm}{listing}
\renewcommand{\seQuerverweisGleichung}{equation}
\renewcommand{\seQuerverweisAlgorithmus}{algorithm}
%
\renewcommand{\seQuerverweisChapter}{chapter}
\renewcommand{\seQuerverweisSection}{chapter}
\renewcommand{\seQuerverweisSubsection}{chapter}
\renewcommand{\seQuerverweisSubsubsection}{chapter}
\renewcommand{\seQuerverweisParagraph}{chapter}


%
% Kommandos f\"ur die Konfiguration von URL-Eintr\"agen im Literaturverzeichnis
%
\renewcommand*{\biburlprefix}{\jblangle{}URL: }
\renewcommand*{\biburlsuffix}{\jbrangle{}}
\renewcommand*{\bibbudcsep}{ -- }
\AddTo\bibsenglish{\renewcommand*{\urldatecomment}{visited on }}

