\section{Abbildungen, Tabellen und Programmlistings\label{gleitobjekte}}

Ein Rechteck besitzt die in \vref{abb1} dargestellte Struktur.

\begin{figure}[htbp]
\centering
\setlength{\unitlength}{1mm}
\begin{picture}(100,30)
\put(0,0){\framebox(100,30){Ich bin kein Quadrat!}}
\end{picture}
\caption[Die Darstellung eines Rechtecks]{Die Darstellung eines Rechtecks\label{abb1}\footnotemark}
\end{figure}
%\footnotetext{\seCite{Vgl.}{S. 400}{The:WA}. Achtung: Dieser Literaturverweis ist  rein fiktiver Natur, 
%die Seite 400 existiert in \seCite{}{}{The:WA} nicht!}\label{fussnote}

Der optionale Parameter im folgenden \verb+\caption+-Kommando
\footnotetext{\seCite{Vgl.}{S. 400}{The:WA}. Achtung: Dieser Literaturverweis ist  rein fiktiver Natur, 
die Seite 400 existiert in \seCite{}{}{The:WA} nicht!}\label{fussnote}


\vspace*{-\baselineskip}
\begin{verbatim}
\caption[Die Darstellung eines Rechtecks]%
{Die Darstellung eines Rechtecks\label{abb1}\footnotemark}
\end{verbatim}
\vspace*{-\baselineskip}

definiert den Eintrag f\"ur das Abbildungsverzeichnis. Dort sollte die Fu{\ss}notennummer nicht auftauchen.
Nutzt man den optionalen Parameter nicht, ist es notwendig,  vor \verb+\footnotemark+ noch ein \verb+\protect+ 
einzuf\"ugen, da \LaTeX{} andernfalls die \"Ubersetzung mit einer Fehlermeldung abbricht. 

Eine Notentabelle kann wie in \vref{noten} dargestellt aussehen.

\begin{table}[htbp]%
\centering%
\begin{tabular}{| c | c |}
\hline
Matrikelnummer & Note \\
\hline
\hline
1234567 & 2,7 \\
\hline
2323456 & 3,5 \\
\hline
9865783 & 1,0 \\
\hline
\end{tabular} 
\caption{Ergebnisse der Klausur Programmierung I\label{noten}}
\end{table}


Eines der wichtigsten Java-Programme \"uberhaupt ist in \vref{hello} zu sehen.

\begin{programm}[htbp]
\begin{lstlisting}
public class HelloDHBW {
  public static void main ( String[] args ) {
    System.out.println ( "Hello DHBW" );
  } // main
} // HelloDHBW
\end{lstlisting}
\caption{Die Klasse \texttt{HelloDHBW}\label{hello}}
\end{programm}

