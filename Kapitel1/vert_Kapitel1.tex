\chapter{Ber\"uhrte Rechtsgebiete}
Gesetzliche Regelungen zum Thema Cloud-Computing befinden sind in unterschiedlichen Rechtsgebieten. Dies r\"uhrt unter anderem daher, da Cloud-Computing auf dem IT-Recht beruht, was selbst eine Vielzahl unterschiedlicher Rechtsnormen tangiert.
Die wichtigsten Rechtsgebiete die f\"ur Cloud-Computing relevant sind, sind:
 \begin{seList}      
\item Zivilrecht
\item  \gls{glos:EG}-Normen
\item Urheberrecht
\item Datenschutzrecht
\item Aktienrecht
\item Handelsrecht
\item Steuerrecht
\item Telekommunikationsrecht
\item Strafrecht
\item \gls{glos:UW}-Recht
 \end{seList}      
Im Folgenden werden die Rechtsgebiete und Normierungen, die f\"ur die weitere Er\"orterung wichtig sind, grundlegend beschrieben und erkl\"art.

\section{Zivilrechtliche Grundlagen}
\subsection{Vertragsarten \& Vertragsschluss}
Zivilrechtlich l\"asst sich jeder Vertrag nach §§145 ff. BGB auf eine Vertragsart oder eine Kombination Mehrerer zur\"uckf\"uhren. Dabei wird der Vertragsschluss unabh\"angig von der Vertragsart durch zwei oder mehrere \"ubereinstimmende Willenserkl\"arungen vollzogen. Die wichtigsten Vertragsarten sind:\footnote{vgl.  §§145 ff. BGB}
 \begin{seList}                            
\item Kaufvertrag 
\item Werkvertrag
\item Dienstvertrag
\item Mietvertrag
\item Leihvertrag
\end{seList} 
\label{vertragsarten_u_vertragsschluss}

\subsection{Mietvertrag}
Im Rahmen des Cloud-Computing ist der Mietvertrag die wichtigste Vertragsart.\newline
Nach §535 BGB verpflichtet sich der Vermieter dem Mieter die Mietsache zu \"uberlassen. Diese muss, gem\"a{\ss} dem Vertrag, in einem gebrauchsf\"ahigen Zustand f\"ur die komplette Mietzeit \"uberlassen werden. Ist dies nicht der Fall, so kann der Mieter Schadens- und Aufwendungsersatz nach §§536 ff. BGB geltend machen.\footnote{vgl.  §§535 ff. BGB}
  \label{mietvertrag}

\section{EG Grundlage: Rom1}
Das auf vertragliche Schuldverh\"altnis anzuwendende Recht wird in der  {\glqq}Verordung (EG) Nr. 593/2008 des europ\"aischen Parlaments und des Rates vom 17. Juni 2008 (Rom1){\grqq} geregelt. Sie besagt, dass aus Gr\"unden des reibungslosen Binnenhandels und vereinfachte Anerkennung von Rechtsentscheidungen den Vertragspartnern eine freie Rechtswahl einger\"aumt wird:{\glqq} Der Vertrag unterliegt dem von den Parteien gew\"ahlten Recht. […]{\grqq}\footnote{Art.3 Abs.1 Rom1}
\newline
{\glqq}Die Parteien k\"onnen jederzeit vereinbaren, dass der Vertrag nach einem anderen Recht zu beurteilen ist als dem, das zuvor entweder aufgrund einer fr\"uheren Rechtswahl nach diesem Artikel oder aufgrund anderer Vorschriften dieser Verordnung f\"ur ihn ma{\ss}gebend war. […]{\grqq}\footnote{Art.3 Abs.2 Rom1}\newline

Die Parteien haben also die Wahl, welches nationale Recht sie anwenden m\"ochten. Dabei gilt aber, dass sich die Vertragspartner \"uber die Rechtswahl einig sind. Haben die Parteien keine Rechtswahl vorgenommen, so unterliegen Dienstleistungsvertr\"age {\glqq}[…]dem Recht des Staates, in dem der Dienstleister seinen gew\"ohnlichen Aufenthalt hat.{\grqq}\footnote{Art.4 Abs.1 b Rom1}

\section{Urheberrecht}
\subsection{Territorialit\"ats- \& Schutzlandprinzip}
Bei der Betrachtung des Urheberrechts findet das Territorialit\"atsprinzip Anwendung. Es besagt, dass das Recht des nationalen Staates zur Beurteilung einer Urheberrechtverletzung angewandt wird.\newline

Das Schutzlandprinzip baut auf dem Territorialit\"atsprinzip auf und besagt, dass das geistige Eigentum nach dem nationalen Recht des Eigent\"umers gesch\"utzt wird.\footnote{vgl. Art.8 Abs.1. Rom1} Alle einseitigen Rechtsgesch\"afte sind demnach nach dem Schutzlandprinzip zu beurteilen. Dabei gilt nach Art.8 Abs. 3 Rom2, dass keine davon abweichenden Regelungen getroffen werden d\"urfen.    
   \label{territorialitaets_u_schutzlandprinzip}

\subsection{Gesch\"utze Werke}
Nach §2 Abs.1 Nr.1 \gls{glos:UrhG} sind Computerprogramme als Sprechwerke im Rahmen des UrhG sch\"utzenswert. Somit gilt das unerlaubte Kopieren eines Programmes als Urheberrechtverletzung.
   \label{geschuetzte_werke}

\subsection{Gegenstand des Schutzes}
Nach §69a UrhG sind alle Ausdrucksformen von Computerprogrammen, einschlie{\ss}lich des Entwurfs sch\"utzenswert, aber nicht ihre Grundidee an sich. Nach Ansicht des \gls{glos:EuGH}s sind aber Benutzeroberfl\"achen keine Ausdrucksform von Computerprogrammen und damit nicht nach §59a UrhG zu sch\"utzen. \footnote{vgl. EuGH vom 22. 12. 2010 - C-393/09}
  \label{gegenstand_des_schutzes}

\subsection{Zustimmungsbed\"urftige Handlungen \& Nutzungsrechte}
Der Rechtsinhaber eines Computerprogramms hat §§31 ff. UrhG und nach §69c UrhG das alleinige Recht, \"uber die Nutzung, Vervielf\"altigung, \"Anderung und Ver\"au{\ss}erung seiner Software zu entscheiden. Wird ein Teil der Software ver\"au{\ss}ert, erlischt das Verbreitungsrecht f\"ur diesen Teil, mit Ausnahme des Vermietungsrechts.\newline

Nutzer, die nicht zugleich Urheber sind, ben\"otigen Nutzungsrechte nach §31 Abs.1 UrhG. Dies ist im Allgemeinen als Lizensierung bekannt, wobei der Nutzer, der nicht Urheber ist, eine Lizenz als Nutzungsberechtigung ben\"otigt. Das Nutzungsrecht ist in §§31ff. UrhG beschrieben. Dabei sind Einschr\"ankungen r\"aumlicher, zeitlicher oder inhaltlicher Art m\"oglich. 
 \label{zustimmungsbeduerftige_handlungen_nutzungsrechte}

\subsection{Sicherungskopien}
Nach §69d Abs.2 UrhG darf eine Person mit Nutzungsberechtigung jederzeit eine Sicherungskopie der Software anfertigen, wenn diese f\"ur die Nutzung relevant ist. Dieses Recht darf nicht durch sonstige vertragliche Regelungen in Frage gestellt werden.
 \label{sicherungskopien}
\section{Datenschutzrecht}
Genauso, wie beim Urheberrecht gilt auch beim Datenschutzrecht das Territorialit\"atsprinzip nach §1 BDSG und richtet sich nach dem Ort der Datenverarbeitung. Nach §3 BDSG ist das Datenschutzrecht bei Verarbeitung von Einzelangaben anzuwenden. Es gilt aber nicht f\"ur Daten, bei denen Identit\"at der nat\"urlichen Person nicht rekonstruierbar ist. Genauso sind Daten juristischer Personen nicht durch das Datenschutzrecht gesch\"utzt.\footnote{vgl. §3 BDSG} 

\subsection{Anonymisieren und Pseudonymisieren}
Unter Anonymisieren ist i.S.d. §3 Abs.6 BDSG das ver\"andern der Daten zu verstehen, mit dem Ziel, das Einzelangaben nicht mehr oder nur mit unverh\"altnism\"a{\ss}ig gro{\ss}em Aufwand einer Person zuordenbar sind.\newline

Das Pseudonymisieren von Daten ist nach §3 Abs.6a BDSG {\glqq}das Ersetzen des Namens und anderer Identifikationsmerkmale durch ein Kennzeichen, zu dem Zweck, die Bestimmung des Betroffenen auszuschlie{\ss}en oder wesentlich zu erschweren.[…]{\grqq}\footnote{§3 Abs.6a BDSG} 
 \label{pseudonymisieren}
\subsection{Erlaubnistatbestand}
Jede Verarbeitung von personenbezogenen Daten muss von einem Erlaubnistatbestand gedeckt sein. Ein Erlaubnistatbestand kann eine Einwilligung des Betroffenen, eine gesetzliche Regelung des BDSG oder eine andere Rechtsvorschrift sein.\footnote{§4 BDSG}
 \label{erlaubnistatbestand}
\subsection{Auftragsdatenverarbeitung}
Der Auftragsdatenverarbeiter wird auf Weisung des Auftraggebers t\"atig und darf damit im Rahmen des §11 BDSG Daten erheben, verarbeiten oder nutzen. Dabei ist der Auftraggeber f\"ur die Einhaltung der Kriterien des §11 BDSG verantwortlich. Der Auftrag ist schriftlich zu erteilen und muss folgende Kriterien regeln:{\glqq} […]
 \begin{seList}                            

\item 1. der Gegenstand und die Dauer des Auftrags,
\item 2. der Umfang, die Art und der Zweck der vorgesehenen Erhebung, Verarbeitung oder Nutzung von Daten, die Art der Daten und der Kreis der Betroffenen,
\item 3. die nach § 9 zu treffenden technischen und organisatorischen Ma{\ss}nahmen,
\item 4. die Berichtigung, L\"oschung und Sperrung von Daten,
\item 5. die nach Absatz 4 bestehenden Pflichten des Auftragnehmers, insbesondere die von ihm vorzunehmenden Kontrollen,
\item 6. die etwaige Berechtigung zur Begr\"undung von Unterauftragsverh\"altnissen,
\item 7. die Kontrollrechte des Auftraggebers und die entsprechenden Duldungs- und Mitwirkungspflichten des Auftragnehmers,
\item 8. mitzuteilende Verst\"o{\ss}e des Auftragnehmers oder der bei ihm besch\"aftigten Personen gegen Vorschriften zum Schutz personenbezogener Daten oder gegen die im Auftrag getroffenen Festlegungen,
\item 9. der Umfang der Weisungsbefugnisse, die sich der Auftraggeber gegen\"uber dem Auftragnehmer vorbeh\"alt,
\item 10. die R\"uckgabe \"uberlassener Datentr\"ager und die L\"oschung beim Auftragnehmer gespeicherter Daten nach Beendigung des Auftrags. 
\end{seList}

[…]{\grqq}\footnote{§11 Abs.1 BDSG}
 \label{auftragsdatenverarbeitung}
\subsection{Technische- \& organisatorische Ma{\ss}nahmen}
Auftragsverarbeitende Stellen m\"ussen technische und organisatorische Ma{\ss}nahmen nach §9 BDSG vornehmen um den Schutz der Daten gem\"a{\ss} des BDSG zu gew\"ahrleisten. Diese Ma{\ss}nahmen m\"ussen nur dann durchgef\"uhrt werden, wenn sie dem Schutzzweck in angemessenem Ma{\ss}e dienen.\newline

Die Anlage zu §9 BDSG sieht f\"ur automatisiert verarbeitete Daten folgende Ma{\ss}nahmen vor, wobei diese auf dem neuesten Stand der Technik durchzuf\"uhren sind:\footnote{vgl. §9 BDSG}
\begin{seList}

\item Zutrittskontrolle
\item Zugangskontrolle
\item Zugriffskontrolle
\item Weitergabekontrolle
\item Eingabekontrolle
\item Auftragskontrolle
\item Verf\"ugbarkeitskontrolle
\item Datentrennung nach Zweck
\end{seList} 
 \label{technische_u_organisatorische_massnahmen}
\section{Aktiengesetz: IT-Sicherheit}
Nach §91 AktG ist der Vorstand einer AG dazu verpflichtet, {\glqq}geeignete Ma{\ss}nahmen zu treffen, insbesondere ein \"Uberwachungssystem einzurichten, damit den Fortbestand der Gesellschaft gef\"ahrdende Entwicklungen fr\"uh erkannt werden.{\grqq}\footnote{§91 AktG} Dies beinhaltet auch IT-Sicherheit herzustellen. \"Ahnliche Regelungen findet man bei anderen Rechtsformen.

 \label{aktiengesetz_it_sicherheit}
\section{Handels- \& Steuerrecht: Aufbewahrungspflicht}
§146 Abs.2 AO besagt unter anderem, dass die Steuerunterlagen in Deutschland aufzubewahren sind. Dies ist daher von N\"oten, da die deutschen Finanzbeh\"orden jederzeit die M\"oglichkeit der Einsicht in Anspruch nehmen k\"onnen. \seFootcite{vgl.}{S.86-87}{taschenbuch}\newline

Nach §§238, 257 HGB ist jeder Kaufmann dazu verpflichtet, seine Handelsb\"ucher, Handelsbriefe und Belege f\"ur Buchungen zehn Jahre lang aufzubewahren. Die Aufbewahrung kann dabei ausgedruckt oder elektronisch erfolgen.
 \label{handels_u_steuerrecht_aufbewahrungspflicht}
\section{Telekommunikationsrecht: Fernmeldegeheimnis}
Dem Fernmeldegeheimnis unterliegen die Inhalte der Kommunikation und deren Umst\"ande. Der Dienstanbieter ist verpflichtet das Fernmeldegeheimnis auch \"uber das Vertragsende hinaus zu wahren. Au{\ss}erdem darf er sich keine Kenntnis \"uber die Inhalte der Telekommunikation beschaffen, es sein denn eine Gesetzesvorschrift verlangt dies.\footnote{vgl. §88 TKG}
 \label{telekommunikationsrecht_fernmeldegeheimnis}
\section{Strafrecht \& Unlautererwettbewerbsgesetz: Geheimnisverrat}
Nach §§ 203, 353, 355 StGB und §17 UWG sind Privat- , Dienst-, Steuer- und Gesch\"aftsgeheimnisse zu wahren. Dies bedeutet, dass ein Mitarbeiter eines Unternehmens dazu verpflichtet ist, die ihm anvertrauten Geheimnisse aus den unterschiedlichen Bereichen zu wahren und nicht an Dritte weiter zu geben. Dabei setzen §§ 203, 355 StGB und §17 UWG eine konkrete Absicht voraus. Lediglich die Verletzung des Dienstgeheimnisses durch die fahrl\"assige Gef\"ahrdung wichtiger \"offentlicher Interessen nach §353 StGB sieht eine Bestrafung fahrl\"assigen Handelns vor. \footnote{vgl. §§ 203, 353, 355 StGB, §17 UWG}
 \label{strafrecht_u_unlautererwettbewerbsgesetz_geheimnisverrat}

  