\chapter{Weitergabe von Software und digitaler Medien sowie deren Veränderung}
\section{Software und digitale Medien}
Als Software und digitales Medium ist im Rahmen dieses Kapitel alles zu verstehen, was auf einer digitalen Plattform wiedergegeben werden kann und von einem Verkäufer an einen Kunden verkauft wurde.
Beispiele für ein digitales Medium sind eine CD, DVD, Diskette, USB-Stick oder eine Musikdatei, die heruntergeladen wurde.\par
Sowohl bei der Software als auch bei digitalen Medien gelten die Regelungen für das Urheberrecht wie in Kapitel 2.3 erläutert.
Bei der Betrachtung von digitalen Medien an sich sind die Rechte und Pflichten zum Thema Datenschutz unter Umständern nur bedingt anwendbar.
So ist bei einer Audiodatei der Datenschutz nicht von belang, bei einer Software zur Überwachung von Netzwerkaktivitäten jedoch schon.
In diesem Kapitel wird jedoch der Schwerpunkt auf das Urheberrecht gelegt.
Somit ist es irrelevant, dass digitale Medien und Software in Bezug auf den Datenschutz unterschiedlich sind, da sich die Ausarbeitung explizit mit den Gemeinsamkeiten der beiden Gebiete beschäftigt.
\section{Veränderungen von Software}
Die Veränderung von Software ist grundsätzlich verboten, da sie gegen das Urheberrecht verstößt.
In bestimmten Ausnahmen ist eine Modifikation seitens der Kunden jedoch ausdrücklich erwünscht und in den jeweiligen Lizenzen verankert.
Die SAP SE bietet ihren Kunden mit der Lizenz zur Nutzung der Software auch gleichzeitig das Recht, diese Software zu verändern, also eine Modifikation vorzunehmen.
Im Rahmen der Geschäftssoftware ist dies auch nötig, um eine möglichst hohe Flexibilität für den Kunden zu erreichen.
Somit können auch individuelle Usecases abgebildet und implementiert werden.\par
Die Kernfrage ist hierbei jedoch, wem diese Modifikationen gehören.\par
Das geistige Eigentum der Modifikation liegt zweifelsfrei beim Kunden der Software, der die Modifikation durchgeführt hat.
Jedoch ist diese Modifikation im System des Softwareherstellers implementiert.
Hierauf liegt ein Urheberrecht.
Im Rahmen der Lizenzvereinbarung kann ein Recht auf die Modifikation eingeräumt werden.
Somit liegt hier kein Bruch des Urheberrechtes vor.
Die Modifikation ist hierbei nicht automatisch vom Urheberrecht geschützt.
In den seltensten Fällen wird das jedoch problematisch, da solche Modifikation nur Firmenintern vorgenommen werden, ohne dabei nach außen zu gehen.
Ein Konflikt mit anderen Rechteinhabern ist somit also unwahrscheinlich.\par
Jedoch ist es auch möglich diese Modifikation im Rahmen einer Lizenz zu schützen und in Absprache mit dem ursprünglichen Hersteller der Software kommerziell zu vertreiben.
\section{Weitergabe von Software}
Die Weitergabe von Software kann im Rahmen der Lizenz untersagt werden\footnote{Dies urteilte das Landgericht München I in Az. 7 O 23237\/05}.
Jedoch urteilte der \gls{glos:EuGH} im Juli 2012 \footnote{Aktenzeichen: C-128/11 }, dass die Weitergabe von Software prinzipiell erlaubt ist und lediglich eine Zustimmung durch den Urheber benötigt.
Der Urheber kann diese Zustimmung nur in wenigen Fällen verweigern, womit in den meisten Fällen einer Weitergabe nicht mehr verwendeter Software nichts im Weg steht.\par
Trotzdem ist die neue Ausgabe von neuen Lizenzen alleiniges Recht vom Urheber beziehungsweise dessen ernannten Vertretern.
Kauft also ein Unternehmen A eine Lizenz vom Hersteller H, so kann diese Lizenz, also das Nutzungsrecht an der Software, von Unternehmen A an Unternehmen B weitergegeben werden.
Jedoch erlischt dann das Nutzungsrecht von Unternehmen A die Software zu nutzen.
Vor der Weitergabe von Software durch Unternehm A sind also genauso viele Lizenzen im Umlauf wie nach der Weitergabe.
Die Weitergabe kann hierbei prinzipiell genauso ausgestaltet sein, wie die Bereitstellung vom Hersteller direkt.
Es sind also verschiedene Geschäftsmodelle möglich, wie zum Beispiel das Vermieten der Software oder das bereitstellen der Software in der Cloud oder die Installation der Software beim Kunden und die Wartung durch den Verkäufer, der hier nicht zwangsläufig der Hersteller ist.\par
Da Unternehmen B keinen direkten Vertrag mit Hersteller H eingegangen ist, können auch keine Ansprüche geltend gemacht werden.
Diese können nur dem Unternehmen A gegenüber geltend gemacht werden, welches sich wiederum mit dem Hersteller auseinandersetzen kann.
Aus diesem Grund, liegt das Interesse des Unternehmens A darin, den Hersteller H als direkten Vertragspartner zu gewinnen, um keine Schadensersatzansprüche erfüllen zu müssen oder bei jedem Problem des Unternehmens B mit involviert zu sein.