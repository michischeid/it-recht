\chapter{Einleitung}
In der hoch digitalisierten Welt, in der wir heute leben, wird der Ruf nach einfacheren Lösungen immer lauter.
Hierzu zählt, dass ein Unternehmen sich nicht mehr um die Instandhaltung von IT-Landschaften kümmern möchte, um Kosten zu sparen.
An diesem Punkt setzt das Cloudcomputing an, wo ein Dienstleister eine IT-Systemlandschaft bereitstellt und ein Unternehmen diese Landschaft nutzen kann.\par
Das Geschäft mit Cloudcomputing hat in der letzten Zeit stark zugenommen und wird auch in Zukunft noch stark steigen\seFootcite{vgl.}{}{wwwStatista}.
Cloudcomputing ist jedoch nur der momentane Hochpunkt in dem Trend immer mehr Kosten für IT einzusparen.
Die Kosten für die Erstellung der verwendeten Software werden seit der Einführung von Standardsoftware permanent reduziert.\par
Wurde vor Einführung der Standardsoftware für jeden Anwendungsfall eine eigene Software geschrieben, so war es mit der Standardsoftware möglich die Kosten für die Erstellung zu reduzieren.
Die Vermietung von Software und der Wartung dieser, anstelle des Kaufes war der nächste Schritt.
Zunächst wurde die Software noch beim Kunden installiert und auch dort gewartet ({\glqq}onPremise{\grqq}), nach und nach wurde die Software allerdings zunehmend in Rechenzentren gehostet.
Im Rahmen der Installation auf nicht Kundeneigenen Systemen, sondern Systemen des Herstellers, redet man vom {\glqq}\gls{glos:SaaS}{\grqq}.
Da die Software nun nicht mehr im unmittelbaren Einzugsbereich des Käufers von Software liegt, sondern in der Cloud, spricht man hier auch vom Cloudcomputing.
Das Cloudcomputing umfasst im Rahmen dieser Arbeit jedoch auch mehr als das reine Nutzungsrecht von Software, nämlich auch die Sicherung der Daten und die Verfügbarkeit der Systeme, um nur einige zu nennen.\par
Im Rahmen dieser Arbeit werden die rechtlichen Grundlagen zum Thema Cloudcomputing erläutert und diskutiert.
Der Schwerpunkt liegt hierbei auf dem Verkauf von Leistungen in der Cloud und deren gesetzlichen Regelungen.
Des Weiteren wird erklärt, wie die Rechtslage bei der Weitergabe von Software ist.