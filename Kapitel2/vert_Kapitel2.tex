\chapter{Vermietung von Software}
\section{Plichten und Rechte der Vertragsparteien}
Grundlegen handelt es sich bei dem Handel mit Cloud-Produkten um Rechtsgesch\"afte, die eines Vertrags bed\"urfen. Allerding existiert kein {\glqq}Cloud-Vertrag{\grqq} als Rechtsgrundlage. Dagegen gibt es einige Normen, die die grundlegenden rechtlichen Aspekte eines Cloud-Produkts regeln. Dar\"uber hinaus gibt es zus\"atzliche M\"oglichkeiten einen {\glqq}Cloud-Vertrag{\grqq} n\"aher zu spezifizieren. Daher werden im folgenden Kapitel die Einzelheiten eines {\glqq}Cloud-Vertrags{\grqq} er\"ortert.

  \subsection{Gew\"ahrleistung \& Nutzungsausfall}
Wie schon in \vref{vertragsarten_u_vertragsschluss} beschrieben basiert jeder Vertrag auf einer der grundlegenden Vertragsarten oder einer Kombination dieser. Somit muss auch der {\glqq}Cloud-Vertrag{\grqq} auf die grundlegenden Vertragsarten zur\"uckzuf\"uhren sein.  Ein {\glqq}Cloud-Vertrag{\grqq} wei{\ss}t Bestandteile unterschiedlicher Vertragstypen auf. So l\"asst sich ein {\glqq}Cloud-Vertrag{\grqq} auf einen Dienstleistungsvertrag zur\"uckf\"uhren, da es sich grundlegend um eine IT-Dienstleistung handelt. Genauso k\"onnte man den {\glqq}Cloud-Vertrag{\grqq} als Werkvertrag ansehen, da bei den meisten Cloud-Produkten ein konkreter Erfolg sichtbar wird. Daneben ist auch ein Leihvertrag bei kostenlosen Cloud-Produkten denkbar. Nach dem BGH ist der ASP-Vertrag grunds\"atzlich als Mietvertrag anzusehen, wobei der ASP-Vertrag dem {\glqq}Cloud-Vertrag{\grqq} sehr nahe kommt.\footnote{vgl. BGH, Urteil vom 15. 11. 2006 - XII ZR 120/04; LG M\"uhlhausen} 
Diese Entscheidung bringt diverse Folgen mit sich:\newline
So muss der Vermieter dem Mieter die Sache \"uber den ganzen Mietzeitraum gebrauchsf\"ahig zur Verf\"ugung stellen. \"Ubertragen auf ein \gls{glos:SaaS}-Produkt hei{\ss}t das, dass es zu 100% der Zeit verf\"ugbar sein m\"usste, was technisch fast unm\"oglich ist. Daher ist es wichtig die Verf\"ugbarkeit zus\"atzlich vertraglich zu regeln. Dies wird normalerweise innerhalb eines \gls{glos:SLA}s vorgenommen. Au{\ss}erdem sind etwaige Fehler in der Software problematisch, da diese eine Pflichtverletzung i.S.d. §280 BGB hervorrufen und der Mieter somit Schadenersatzanspr\"uche geltend machen kann. Auch hier ist es Sinnvoll eine Haftungseinschr\"ankung im Rahmen des SLAs zu definieren.
\subsection{Anwendbares Recht}
Das anzuwendende Recht innerhalb der EU, ist wie in \vref{territorialitaets_u_schutzlandprinzip} beschrieben, von den Vertragsparteien frei w\"ahlbar. Das hei{\ss}t, dass sich z.\,B. auch zwei deutsche Unternehmen auf die Anwendung des britischen Rechts einigen k\"onnen. Ist das anwendbare Recht nicht vertraglich geregelt worden, wird bei Dienstleistungsvertr\"agen das nationale Recht angewandt, in dem der Dienstleister seinen Sitz hat. Somit ist bei einem {\glqq}Cloud-Vertrag{\grqq} das anwendbare Recht, sofern es nicht anders geregelt ist,  das nationale Recht des Cloud-Anbieters. Wichtig dabei ist, dass es nicht darauf ankommt, wo letztlich die Server des Anbieters stehen. Gerade bei einem Cloud-Produkt w\"are m\"oglicherweise der Standort des Servers gar nicht vorhersehbar und damit unm\"oglich rechtlich zu erfassen.\newline
Wird der {\glqq}Cloud-Vertrag{\grqq} gegen\"uber einem Verbraucher geschlossen ergibt sich eine andere Regelung. Hier wird das nationale Recht des Verbrauchers herangezogen. Wird mit einem Verbraucher als Vertragspartner eine Rechtswahl getroffen, darf diese nach dem G\"unstiger-Grundsatz nicht zur Schlechterstellung des Verbrauchers f\"uhren.\seFootcite{vgl.}{S.28-29}{taschenbuch}
Das anwendbare Recht f\"ur urheberrechtliche Fragestellungen folgt einem anderen Prinzip: dem Schutzlandprinzip. N\"aheres dazu in \vref{territorialitaets_u_schutzlandprinzip}.\newline
Die Auswirkungen des Schutzlandprinzips auf den  Cloud-Anbieter ergeben sich folgender Ma{\ss}en:\newline
Ist das Cloud-Produkt international zug\"anglich kann der Cloud-Anbieter die Rechtsverletzung in einem EU-Land nach seiner Wahl gelten machen. Ist allerdings die Verf\"ugbarkeit des Verletzer-Produktes geografisch steuerbar, so muss der Cloud-Anbieter in jedem Land gegen den Verletzer vorgehen. \newline
Zudem ist auch der umgekehrte Fall denkbar: Hier w\"urde der Cloud-Anbieter selbst zum Verletzer des Urheberrechts eines Dritten. Hier kann der Cloud-Anbieter in jedem Land f\"ur eine Rechtsverletzung eines Dritten rechtlich belangt werden. Abhilfe der Problematik des Schutzlandprinzips l\"asst sich bei Cloud-Produkten nicht zu 100% erreichen. Der Cloud-Anbieter kann aber das Risiko einer Verurteilung durch eine Rechtsverletzung gegen\"uber einem Dritten in einem beliebigen Land mindern. Eine M\"oglichkeit ist, die urheberrechtsrelevanten Handlungen auf die eigene Infrastruktur zu minimieren. Das hei{\ss}t, der Cloud-Anbieter sollte auf die Ver\"offentlichung von Apps verzichten und lediglich eine international verf\"ugbare Webanwendung bereitstellen. Die ledigliche M\"oglichkeit des Zugriffs auf eine Webanwendung wird von den meisten Gerichten nicht als urheberrechtlich relevante Handlung empfunden. \newline
Des Weiteren ist Einrichtung einer Geo-Sperre sinnvoll. Hier erm\"oglicht der Cloud-Anbieter nur Zugriffe bestimmte L\"ander und geht somit nicht das Risiko ein, f\"ur eine Rechtsverletzung in einem beliebigen Land haften zu m\"ussen. Allerdings gibt es bei Geo-Sperren die technische M\"oglichkeit der Umgehung dieser, womit die Nicht-Verf\"ugbarkeit in einem Land nicht unbedingt gew\"ahrleistet ist.\seFootcite{vgl.}{ S.262-265}{handbuch}

 \subsection{Softwareschutz i.S.d. Urheberrechts}
Computerprogramme gelten i.S.d. Urheberrecht als Sprechwerk und damit als sch\"utzenswert.(siehe {ver}) Bei Cloud-Produkten ist es aber aus Performancegr\"unden oftmals gegeben, Datenbankteile oder andere Programmteile auf verschiedene Server verteilt werden. Dabei stellt sich die Frage ob Programmteile i.S.d. Urheberrechts als sch\"utzenswert sind. Dazu muss das Programmteil unabh\"angig von Gesamtwerk f\"ur sich i.S.d. §2 UrhG sch\"utzenswert sein. Nach dem OLG Hamburg sind Programmteile nur dann sch\"utzenswert, wenn sie einen wesentlichen Teil des Programms ausmachen.\footnote{vgl. OLG Hamburg, 11.01.2001 - 3 U 120/00} Allerdings finden sich diese Kriterien in keine Rechtsnorm wieder, womit man sich nicht auf die Allgemeing\"ultigkeit dieser Entscheidung verlasen kann.\newline
Nach §69a UrhG sind Programme einschlie{\ss}lich ihres Entwurfs i.S.d. Urheberrechts sch\"utzenswert(siehe {ver}). Die Benutzeroberfl\"ache ist dabei die Ausnahme und gilt in der Regel als nicht sch\"utzenswert. Somit stellt eine m\"oglicher Weise als Vervielf\"altigung angesehen Speicherung der Benutzeroberfl\"ache im Cache des Clients keine Urheberrechtsverletzung dar.\newline
Bei SaaS gilt, dass Nutzer, die nicht zugleich Urheber sind, Nutzungsrechte f\"ur die Nutzung der Software ben\"otigen. Die Nutzungenrechte sind Lizenzen, die Kunde vom SaaS-Anbieter kauft.   

    \subsection{Anwendbares Datenschutzrecht}
Auch im Datenschutz findet das Territorialit\"atsprinzip Anwendung. Es richtet sich hierbei nach dem Ort der Datenverarbeitung. Beim Cloudcomputing sind das die einzelnen Server auf denen die Daten verarbeitet werden. Unklar ist aber oft bei Cloud-Computing von welchem Server gerade die Daten kommen. Die Daten k\"onnen verteilt auf unterschiedlichen Servern, wom\"oglich \"uber L\"andergrenzen hinweg gespeichert werden. Daher wird bei Cloud-Computing der Sitz des Auftraggebers der Datenerhebung herangezogen. Es ist also letztendlich das nationale Recht des Kunden bei datenschutzrechtlichen Fragestellungen Ma{\ss}gebend.

    \subsection{Schutzf\"ahige Daten}
Grunds\"atzlich sch\"utzt das Datenschutzrecht jegliche Art personenbezogener Daten. Dabei muss es sich um Daten einer nat\"urlichen Person handeln. Diese muss dabei bestimmbar sein. Typische Daten im Cloud-Computing sind Daten \"uber das Nutzungsverhalten der Software, IP-Adressen und Logindaten. F\"ur Anbieter von Cloud-Produkten ist es daher wichtig personenbezogene Daten soweit zu ver\"andern, dass die Person dahinter nicht mehr bestimmbar ist. Das Anonymisieren und Pseudonymisieren von Daten erm\"oglicht dies (siehe \vref{pseudonymisieren}). 
Nach §3 Abs.6 BDSG soll der Aufwand, die Identit\"at zu rekonstruieren verh\"altnism\"a{\ss}ig sein. Dabei ist es streitig, was verh\"altnism\"a{\ss}ig bedeutet. Es ist bei Cloud-Produkten technisch m\"oglich personenbezogene Daten durch anonymisierte Zeichenfolgen zu ersetzen, bei denen Rekonstruktion der Identit\"at unverh\"altnism\"a{\ss}ig schwer w\"are. Allerding k\"onnen Nutzungsdaten nicht unwiderruflich verschl\"usselt werden, wenn diese in Zukunft angezeigt oder ausgewertet werden m\"ussen. Pseudonymisieren, bei dem die Daten durch Pseudonyme ersetzt werden, stellt keine vollst\"andige Anonymit\"at her. Denn, derjenige der die Ersetzungsregel kennt, kann ohne gro{\ss}en Aufwand die personenbezogenen Daten rekonstruieren. Daher ist bei pseudonymisierten Daten immer zu entscheiden, f\"ur wen die Daten anonym sind und f\"ur wen nicht. Gleiches gilt f\"ur verschl\"usselte Daten. F\"ur den Schl\"usselinhaber sind die Daten als personenbezogene Daten zu betrachten, w\"ahrend sie f\"ur andere anonym sind.\newline
Das BDSG schreibt vor, dass jede Verarbeitung personenbezogener Daten von einem Erlaubnistatbestand gedeckt sein muss (siehe \vref{erlaubnistatbestand}). Es bedarf demnach einer Zustimmung des Betroffenen oder einer Rechtsvorschrift, die die Datenverarbeitung erlaubt. Somit m\"usste der Cloud-Anbieter \"uber eine Einwilligung aller Betroffen verf\"ugen. Eine Abhilfe stellt die Auftragsdatenverarbeitung dar. Hierbei handelt der Cloud-Anbieter im Auftrag des Cloud-Abnehmers. Damit kann der Cloud-Anbieter im selben Umfang Daten verarbeiten wie der Cloud-Abnehmer. Somit ist der Cloud-Abnehmer f\"ur die Einhaltung des Datenschutzrechts verantwortlich. Dies setzt voraus, dass der Cloud-Abnehmer eine juristische Person ist und es sich damit bei dem Cloud-Produkt um eine B2B-L\"osung handelt.\newline
Die Alternativl\"osung zur Auftragsdatenverarbeitung ist die Funktions\"ubertragung, bei der Cloud-Anbieter und Cloud-Abnehmer eigenst\"andig f\"ur die Einhaltung des Datenschutzrechts verantwortlich sind. Dabei gilt aber, dass der Cloud-Anbieter eigenverantwortlich handeln muss. Eine \"Ubertragung personenbezogener Daten an den Cloud-Anbieter w\"are dann eine \"Ubermittlung von Daten an Dritte. Zur Verarbeitung der Daten durch den Cloud-Anbieter bedarf es einer Einwilligung des Betroffenen.\newline 
Im Regelfall wird es sich bei Cloud-Produkten um eine Auftragsdatenverarbeitung handeln. Verarbeitet der Cloud-Anbieter aber Daten f\"ur eigene Zwecke, tr\"agt er auch die datenschutzrechtliche Verantwortung i.S.d. §11 Abs.1 BDSG. Ein Beispiel bei dem eine Funktions\"ubertragung relevant w\"are die Beauftragung eines Meinungsforschungsinstitutes, bei dem das Institut eigenverantwortlich handelt. 
 \subsection{ADV-Vertrag}
Die Auftragsdatenverarbeitung zwischen Cloud-Anbieter und Cloud-Abnehmer muss Vertraglich ausgestaltet sein (siehe \vref{auftragsdatenverarbeitung}). Die Vertragsbestandteile, wie in \vref{auftragsdatenverarbeitung} beschrieben, sind nach § 11 BDSG zu beachten und m\"ussen zwingend eingehalten werden.\newline
Au{\ss}erdem ist der ADV Vertrag nur dann wirksam, wenn er schriftlich geschlossen wird. Es soll damit sichergestellt werden, dass der Cloud-Anbieter wirklich im Auftrag des Cloud-Abnehmers gehandelt hat. \"Uber die in \vref{auftragsdatenverarbeitung} beschriebenen Inhalte hinaus kann der ADV-Vertrag auch noch durch weitere Inhalte erg\"anzt werden, wie etwa die konkrete technische Ausgestaltung des Datenschutzes.\seFootcite{vgl.}{}{wwwIlex}\newline
Ein wichtiger Bestandteil des ADV-Vertrags sind die technischen und organisatorischen Ma{\ss}nahmen zum Datenschutz. Diese zu treffenden Ma{\ss}nahmen sind in der Anlage zu §9 BDSG genauer beschrieben (siehe \vref{technische_u_organisatorische_massnahmen}).  Diese Ma{\ss}nahmen werden im Folgenden genauer erl\"autert:\seFootcite{vgl.}{}{wwwBsi} 

 \begin{seList}                            

\item \textbf{Zutrittskontrolle:}\newline
Der physische Zugang zu den Daten darf nur einem bestimmten Personenkreis zug\"anglich sein. Dies kann z.\,B. durch eine mit Fingerabdruck gesicherte T\"ure zum Rechenzentrum umgesetzt werden.
\item \textbf{Zugangskontrolle:}\newline
Der digitale Zugang zum EDV-System muss kontrolliert sein. Dies kann z.\,B. \"uber ein Login am System erfolgen.
\item \textbf{Zugriffskontrolle:}\newline
Personen die eine Zugangsberechtigung besitzen sollen nur auf die Daten zugreifen k\"onnen, f\"ur die sie eine Berechtigung haben. Dies kann z.\,B. durch eine strikte Umsetzung einer rollenbasierten Benutzerverwaltung umgesetzt werden.  
\item \textbf{Weitergabekontrolle:}\newline
Unbefugte d\"urfen keine Daten lesen, ver\"andern, l\"oschen oder kopieren k\"onnen. Umsetzen l\"asst sich die Weitergabekontrolle z.\,B. durch eine verschl\"usselte Daten\"ubertragung, kombiniert mit einem Login am System.
\item \textbf{Eingabekontrolle:}\newline
Es muss nachvollziehbarsein, wer einen Datensatz erstellt, ge\"andert oder gel\"oscht hat. Durch ein Protokoll auf Datenbankebene k\"onnen diese Informationen persistiert werden. 
\item \textbf{Auftragskontrolle:}\newline
Die Daten m\"ussen gem\"a{\ss} der Weisung des Cloud-Abnehmers verarbeitet werden. Es muss also sichergestellt werden, dass der Cloud-Abnehmer die Datenverarbeitung nachvollziehen kann und damit einverstanden ist.
\item \textbf{Verf\"ugbarkeitskontrolle:}\newline
Die Daten m\"ussen vor h\"oherer Gewalt gesch\"utzt werden. Sinnvoll hierf\"ur w\"aren regelm\"a{\ss}ige Datensicherungen, m\"oglichst an einem anderen Ort.
\item \textbf{Datentrennung nach Zweck:}\newline
Es soll sichergestellt werden, dass die Daten gem\"a{\ss} ihrem Zweck verarbeitet und nicht missbraucht werden. Dies kann durch ein nach Zweck modularisiertes  EDV-System realisiert werden.
\end{seList}

 \subsection{Risiken und Probleme bei der Auslagerung von Unternehmensdaten in die Cloud}
Innerhalb eines {\glqq}Cloud-Vertrags{\grqq} sollte auch zus\"atzliche Kriterien Aufgenommen werden um die IT-Sicherheit wie gew\"unscht herzustellen. So ist z.\,B. der Vorstand einer AG, wie in \vref{aktiengesetz_it_sicherheit} beschrieben, dazu verpflichtet, die IT-Infrastruktur zu abzusichern. Bei einer Auslagerung in die Cloud, muss somit der Vorstand daf\"ur sorgen, dass der Cloud-Anbieter die IT-Sicherheit f\"ur ihn umsetzt. Auch bei anderen Rechtformen finden sich \"ahnliche Regelungen, die alle auch gelten, wenn die Daten in der Cloud liegen.\newline 
Im Steuerrecht findet sich der Regelung, dass Steuerunterlagen im Inland aufzubewahren sind. Auf Bewilligung der Steuerbeh\"orde, k\"onnen die Daten auch auf Servern innerhalb der Europ\"aischen Union gelagert werden, aber unter keinen Umst\"anden au{\ss}erhalb. Somit muss der Cloud-Abnehmer daf\"ur sorgen, dass der Cloud-Anbieter diese Daten auf Serven innerhalb Deutschlands oder bei Genehmigung innerhalb der Europ\"aischen Union ablegt. Au{\ss}erdem m\"ussen die Daten auf Anfrage der unverz\"uglich an die Steuerbeh\"orde weitergeleitet werden. Im Handelsrecht ist Festgelegt, dass Daten wie Belege oder Handelsb\"ucher zehn Jahre lang aufbewahrt werden m\"ussen (siehe \vref{handels_u_steuerrecht_aufbewahrungspflicht}). Auch dies muss in einer {\glqq}Cloud-Vertrag{\grqq} geregelt werden, sodass die Daten innerhalb von zehn Jahren jederzeit abrufbar sind.\seFootcite{vgl.}{S86-87}{taschenbuch}\newline
Verarbeitet die Cloud-Software E-Mails, gilt nach dem TKG, dass der Cloud-Anbieter keinerlei Informationen \"uber den Inhalt, Empf\"anger und Absender auswerten darf. Grundlegendes dazu ist in \vref{telekommunikationsrecht_fernmeldegeheimnis} beschrieben. Aber auch auf Seite des Cloud-Abnehmers darf der Arbeitgeber keinerlei Informationen aus den E-Mails auswerten. Ein Versto{\ss} gegen das Fernmeldegeheimnis wird mit bis zu f\"unf Jahren Freiheitsstrafe geahndet und ist demnach kein Bagatelldelikt.\newline 
Berufsgruppen, wie \"Arzte, Psychologen oder Rechtsanw\"alte sollten sich den Einsatz einer Cloud-Software genau \"uberlegen, da f\"ur diese Berufsgruppen besondere Regelungen gelten. N\"aheres dazu ist in \vref{strafrecht_u_unlautererwettbewerbsgesetz_geheimnisverrat} beschrieben. Sie d\"urfen die ihnen anvertrauten Geheimnisse nicht an Dritte weitergeben. Bei einer Cloud-Software k\"onnte es sein, dass diese Geheimnisse nicht mehr unmittelbar in der Hand der Geheimnistr\"ager sind. Der Geheimnistr\"ager w\"urde sich dann des Geheimnisverrats strafbar machen. Daher ist eine Cloud-Software f\"ur diese Berufsgruppen eher nicht geeignet und wenn doch, nur mit strenger Einhaltung der Anonymisierung, Pseudonymisierung und Verschl\"usselung der Daten.\seFootcite{vgl.}{}{wwwRechtFreundlich}\newline
\"Ahnliches gilt f\"ur Mitarbeiter, die sich strafbar machen k\"onnen, wenn sie Betriebsgeheimnisse in die Cloud stellen. Daher m\"ussen die Mitarbeiter aufgekl\"art werden um keine geheimen Daten unwissentlich in die Cloud zu bringen und somit wom\"oglich an Dritte weiter zu geben.

\section{Service Level Agreement}
Zus\"atzlich zu den gestzlichen Regelungen haben sich haben sich Zusatzregelungen  f\"ur IT-Diestleistungsvertr\"age in der Informationstechnologie entabliert. Solche Zusatzregelungen werden in SLAs zusammengefasst.
Ein SLA ist eine Vereinbarung zwischen dem IT-Dienstleistungsanbieter und dessen Kunde \"uber die Dienstleistungsg\"ute.\newline
Typische Bestandteile eines SLAs sind:
\begin{seList}
\item \textbf{Laufzeit des Vertrags:}\newline
Hier werden der Beginn und das Ende definiert, sowie Vereinbarungen \"uber Verl\"angerungen und Beendigung getroffen.
\item \textbf{Angestrebtes Kundenergebnis:}\newline
Dies beinhaltet den Nutzen f\"ur den Kunden und Gew\"ahrleistungen, wie z.\,B. die Verf\"ugbarkeit des Softwareprodukts.
\item \textbf{Kommunikation zwischen den Vertragsparteien und Verantwortlichkeiten:}\newline	
Es werden die Kontaktpersonen, Verantwortliche, das Berichtswesen und die Ermittlung der Kundenzufriedenheit detailliert aufgef\"uhrt.
\item \textbf{Anforderungen an das Service- und Support-Level:}\newline	
Es kann hier geregelt werden, wo und wie der Support stattfindet, wie z.\,B. Remote oder vor Ort. Au{\ss}erdem werden Reaktion- und L\"osungsfindungszeiten, sowie die Zuverl\"assigkeit, Performance, Wartbarkeit und maximale Ausfallzeiten der Software aufgef\"uhrt.
\item \textbf{Finanzielle Parameter und Sanktionen bei Nichteinhaltung der Pflichten:}\newline	
Dies beinhaltet den Basispreis, sowie Preise f\"ur Zusatzfunktionalit\"aten, aber auch Regelungen zu den Vertragsstrafen, wie z.\,B. Strafen f\"ur die Nichteinhaltung des Zeitrahmens
\end{seList}

\section{Mindestbestandteile des Cloud-Vertrags}
\begin{seList}
\item Bezeichnung der Vertragsparteien
\item Bezeichnung der Software
\item SLA
\item Geschuldeter Erfolg
\item Art und Umfang der Nutzung
\item Entgelt
\item Hauptpflichten
\item Nebenleistungspflichten
\begin{seList}                           
\item Bereitstellung aller erforderlichen Informationen
\item Mitwirkung bei Qualit\"atssicherungsma{\ss}nahmen
\item Ansprechpartner
\item Ggf. Erm\"oglichung einer Migration
\item Bereitstellen von Testdaten
\end{seList}     
\item Haftung/Gew\"ahrleistung
\item Datenschutz und Datensicherheit
\begin{seList}                           
\item Mindestanforderungen nach §11 BDSG
\item Technische - und organisatorische Ma{\ss}nahmen
\item Ggf. Unterauftragsverh\"altnisse
\item Kontrollrechte des Cloud-Abnehmers
\item Weisungsbefugnis des Cloud-Abnehmers
\end{seList}                           
\item Wahl der anzuwendenden nationalen Rechtsordnung und des Gerichtsstands
\item Schriftform

\end{seList}


