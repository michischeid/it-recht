% Konfigurationsdatei f\"ur die Pfaddefinitionen einlesen
\input{se-wa-pfade}
%
%
% Festlegung der Sprache: 
\newcommand{\seWaSprache}{deutsch}
%\newcommand{\seWaSprache}{englisch}

%
% Einlesen der .sty-Dateien
%
\input{\seWaPathSty/se-wa-input-styles-v096}

%
% Individuelle Konfiguration des Dokumentes
%
\input{\seWaPathText/wa-konfiguration}

%
% Definition von Abk\"urzungen, Symbolen und eventuell Glossareintr\"agen
%
\seNewAcronymGlossaryEntry[EG]{glos:EG}{EG}{Europ\"aische Gemeinschaft}{EG}{Die EG ist eine internationale Gemeinschaft innerhalb der Europ\"aischen Union, die unter anderem Normen f\"ur ein gemeinsames Recht beschlie{\ss}t.}
\seNewAcronymGlossaryEntry[UW]{glos:UW}{UW}{Unlauterer Wettbewerb}{UW}{Unlauterer Wettbewerb ist ein Versto{\ss} gegen die guten Sitten.}
\seNewAcronymGlossaryEntry[EuGH]{glos:EuGH}{EuGH}{Europ\"aischer Gerichtshof}{EuGH}{Der EuGH ist das oberste Gericht der Europ\"aischen Union, mit dem Sitz in Luxemburg.}
\seNewAcronymGlossaryEntry[UrhG]{glos:UrhG}{UrhG}{Urheberrechtsgesetz}{UrhG}{Das UrhG enth\"alt Normen zum Schutz des geistigen Eigentums.}
\seNewAcronymGlossaryEntry[SaaS]{glos:SaaS}{SaaS}{Software-as-a-Service}{Software-as-a-Service}{SaaS ist eine Variante des Cloud-Computing, bei der Software als Dienstleistung angeboten wird.}
\seNewAcronymGlossaryEntry[SLA]{glos:SLA}{SLA}{Service-Level-Agreement}{Service-Level-Agreements}{Ein SLA ist ein Vereinbarung zwischen Auftraggeber und Dienstleister \"uber eine bestimmte Dienstg\"ute, wie z.\,B. Verf\"ugbarkeit der Software.}
\seNewAcronymGlossaryEntry[BDSG]{glos:BDSG}{BDSG}{Bundesdatenschutzgesetz}{BDSG}{Das BDSG regelt auf nationaler Ebene datenschutzrechtliche Fragestellungen.}
\seNewAcronymGlossaryEntry[ASP]{glos:ASP}{ASP}{Applikation-Service-Provider (Anwendungsdienstleister)}{ASP}{Ein ASP stellt eine digitale Dienstleistung \"uber das Internet zur Verf\"ugung. }
\seNewAcronymGlossaryEntry[B2B]{glos:B2B}{B2B}{Business to Business}{}{Bei B2B handelt es sich um eine gesch\"aftliche Beziehung zwischen zwei Unternehmen.}
 

\seIstSeminararbeit{}
%\seIstErsteProjektarbeit{}
%\seIstZweiteProjektarbeit{}
%\seIstBachelorarbeit{}

\newcommand{\version}{0.96}

% 
% Diese Redefinition ist nur f\"ur den Anhang B der  
% Vorlage (Hinweise zur Installation und \"Ubersetzung)
% notwendig; f\"ur Ihre Bachelorarbeit spielt sie keine Rolle
%
\renewcommand{\seVorlage}{\jobname}

%

\begin{document}

% Erzeugung des Titelblatts
%
%
%
\seTitelblattWissenschaftlicheArbeit[
%hilfslinien=ja,
%dhbwlogoSkalierung=0.5,
%dhbwlogoDeltaX=2.4,
%dhbwlogoDeltaY=-10,
firmenlogo=firmenlogo,
firmenlogoSkalierung=0.23,
firmenlogoDeltaX=0,
firmenlogoDeltaY=-16,
studiengang=\seWirtschaftsinformatik,
%studienrichtung=\seApplicationManagement,
%studienrichtung=\seSalesUndConsulting,
studienrichtung=\seSoftwareEngineering,
thema= Erweiterung des Repository Browsers in der Entwicklungsumgebung f\"ur Partner,
verfasser={{Lars Tilsner}, {Michael Scheid}},
%verfasserin=Melanie Musterfrau,
matrikelnummer={{xxxx}, {5102819}},
kurs=WWI\,12\,SE\,A,
firma=SAP AG,
% Da im Text ein Komma enthalten ist, muss der Text eingeklammert werden
%abteilung={ByDesign Dev Partner/Extensibility},
%studiengangsleiterin=,
studiengangsleiter=Prof. Dr.-Ing. J\"org Baumgart,
%wissenschaftlicheBetreuerinName=Dr. Melanie Mustermann,
%wissenschaftlicheBetreuerinEmail=melanie.mustermann@musterfirma.de,
%wissenschaftlicheBetreuerinTelefon=0621/999999,
wissenschaftlicherBetreuerName=Gregor Tielsch,
wissenschaftlicherBetreuerEmail=gregor.tielsch@sap.com,
wissenschaftlicherBetreuerTelefon=06227/748996,
bearbeitungszeitraumVon=26. August 2013,
bearbeitungszeitraumBis=11.November 2013,
sperrvermerk=nein
]


% Erzeugung der englischen Kurzfassung (Abstract); Verfasser, Firma und Thema werden automatisch \"ubernommen
%
% Der optionale Parameter kann verwendet werden, um f\"ur das Thema der Arbeit eine 
% andere Formatierung vorzunehmen; das sollte in der Regel nicht erforderlich sein;
% ausserdem besteht die Gefahr inkonsistenter Titel auf dem Titelblatt und in der 
% Kurzfassung
%
%
% Achtung: Das Kommando erzeugt nur dann eine Ausgabe, wenn \seWaSprache den Wert englisch besitzt
%
%
\seAbstract{} % dieses Kommando sollte standardm\"assig verwendet werden

%\seAbstract[\LaTeX-Vorlage zur Anfertigung \seThemaWaArbeit{} (Version \version{})]



% Erzeugung der Kurzfassung; Verfasser, Firma und Thema werden automatisch \"ubernommen
%
% Der optionale Parameter kann verwendet werden, um f\"ur das Thema der Arbeit eine 
% andere Formatierung vorzunehmen; das sollte in der Regel nicht erforderlich sein;
% ausserdem besteht die Gefahr inkonsistenter Titel auf dem Titelblatt und in der 
% Kurzfassung
%
%\seKurzfassung{} % dieses Kommando sollte standardm\"assig verwendet werden







%\seKurzfassung[\LaTeX-Vorlage zur Anfertigung \seThemaWaArbeit{} (Version \version{})]


% Beispiel f\"ur ein Kapitel, dass vor dem Einleitungskapitel kommt, z. B. ein Vorwort oder eine Danksagung
%\seKapitelVorEinleitung{Vorwort}




% 2012-02-06 Inhaltsverzeichnis muss vor den weiteren Verzeichnisses kommen
%
%
% Ausgabe des Inhaltsverzeichnisses
%
%
\seInhaltsverzeichnis[%
einrueckung=ja,
gliederungsebenen=4
]




% Ausgabe der verschiedenen Verzeichnisse
% abk: Abk\"urzungsverzeichnis
% sym: Symbolverzeichnis
% abb: Abbildungsverzeichnis
% tab: Tabellenverzeichnis
% prg: Listingverzeichnis
% alg: Algorithmenverzeichnis
%
%
% Achtung: Abk\"urzungs- und Symbolverzeichnis werden nur ausgegeben, wenn mindest ein Symbol bzw. 
%                mindestens eine Abk\"urzung in der Arbeit verwendet wurden
%
%
% gliederungsebene:
% -- section: die Verzeichnisse werden einem Kapitel {\glqq}Verzeichnisse{\grqq} untergliedert
% -- chapter: die Verzeichnisse sind jeweils eigene Kapitel
% imInhaltsverzeichnis: ja/nein -- Sollen die Verzeichnisse im Inhaltsverzeichnis enthalten sein?
\seVerzeichnisse[gliederungsebene=section,imInhaltsverzeichnis=ja]{abk}{sym}{abb}{prg}






% Erstes eigentliches Kapitel der Arbeit; typischerweise das Einleitungskapitel;
% hier muss wieder auf die Nummierung mit arabischen Seitenzahlen umgestellt werden
%



% Erstes Hauptkapitel der Arbeit 
%
%
%

% Mit markright kann eine verk\"urzte Version der \"Uberschrift f\"ur den Seitenkopf generiert werden
%
%
%\markright{Formaler Aufbau}



% Anhang der Arbeit
% 
%

\pagenumbering{arabic}
\input{Felix}
\input{\tOOA/LarsVerteiler}
\section{Technische Architektur}
\input{Michael}
\chapter{Tests}
\input{tests}
\input{AspectJ}

  
%\seAppendix{}
%\newcommand{\waInputStyles}{\texttt{se-wa-input-styles-v096.tex}}
\input{\seWaPathText/se-latex-kommandos}

\input{\seWaPathText/se-englisch}

\input{\seWaPathText/se-hinweise-installation}

\input{\seWaPathText/se-nuetzliche-pakete}

\input{\seWaPathText/se-tipps}

\input{\seWaPathText/se-literaturempfehlungen}

\input{\seWaPathText/se-hinweise-literaturverzeichnis}


%
%  Erzeugung eines Glossars
%
% Achtung: Das Glossar wird nur ausgegeben, wenn mindestens ein Eintrag in der Arbeit 
%                definiert wurde
%
%
\newpage
\sePrintGlossary{}


%
% Literaturverzeichnisses
%
%\newpage
\sePrintBibliography{}

%\input{\seWaPathText/se-test-literaturverzeichnis}


%
% Festlegung des grundlegenden Formatierungsstils des Literaturverzeichnis
%
\bibliographystyle{jurabib}

% Eigentliche Ausgabe der in der Arbeit verwendeten Quellen
%
%
% Angabe der bib-Dateien, in denen die Quellen beschrieben sind;
% die Angabe geht davon aus, dass eine wa.bib-Datei in demselben 
% Verzeichnis liegt, wie se-ba-vorlage.tex
%

% 2012-02-06
%
% Umbenennung von Literatur- in Quellenverzeichnis
% 
%\renewcommand*{\bibname}{Quellenverzeichnis}
\seBibliography{wa}


%
% Erzeugung der ehrenw\"ortlichen Erkl\"arung
%
% Der optionale Parameter kann verwendet werden, um f\"ur das Thema der Arbeit eine 
% andere Formatierung vorzunehmen; das sollte in der Regel nicht erforderlich sein;
% ausserdem besteht die Gefahr inkonsistenter Titel auf dem Titelblatt und in der 
% ehrenw\"ortlichen Erkl\"arung
%
\seEhrenwoertlicheErklaerung{} % dieses Kommando sollte standardm\"assig verwendet werden
%\seEhrenwoertlicheErklaerung[\LaTeX-Vorlage zur Anfertigung \seThemaWaArbeit{} (Version \version{})]


\end{document}











