\chapter{Fazit}
Abschließend lässt sich sagen, dass der Trend zur Auslagerung der Software ungebrochen ist.
Immer mehr Unternehmen nehmen Clouddienste in Anspruch und müssen sich mit dem Thema Datenschutz und Urheberrecht auseinandersetzen.
Das Rechtsgebiet rund um die Cloud ist eine Mischung aus den verschiedensten Rechtsgebieten des Zivilrecht, Datenschutzrecht, Urheberrecht und europäischen Recht.\par
Kapitel 2 der Arbeit gibt an, welche Rechtsbereiche tangiert werden, und Kapitel 3 und 4, wie diese Rechtsgebiete einander tangieren.
Im Rahmen der Vermietung und der Weitergabe von Software gelten ähnliche Regelungen wie auch beim Verkauf von Software und der Mietung anderer Gegenstände.\par
Anhand dieser Arbeit ist es möglich ein Grundverständnis für die Rechtsgebiete, sowie deren Wechselwirkung untereinander zu erhalten.
Die wichtigsten Fachbereiche und Regelungen rund um die Cloud wurden dargestellt und die Weitergabe von Software erläutert.
Darüber hinaus bietet die Ausarbeitung eine Möglichkeit der Überprüfung ob Cloudgeschäfte, die abgeschlossen werden sind, den Mindestanforderungen genügen.\par
